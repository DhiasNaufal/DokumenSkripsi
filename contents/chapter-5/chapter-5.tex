\chapter{Kesimpulan dan Saran}

\section{Kesimpulan}

Berdasarkan pengembamgan yang telah dilakukan, diperoleh beberapa kesimpulan sebagai berikut:
\begin{enumerate}
    \item Aplikasi MedQ sebagai sebuah media pembelajaran elektronik yang secara spesifik membahas mengenai sistem Clinical Decision Support System telah berhasil dikembangkan dengan mengimplpementasikan lima aktivitas utama. Aktivitas tersebut dimulai dari melihat dashboard halaman utama, mempelajari materi, mengerjakan kuis singkat mengenai materi, melihat urutan leaderboard dari kuis terkait, dan melihat penghargaan atau \textit{achievement} berdasarkan kuis terkait.
    \item Aplikasi MedQ juga dikembangkan menggunakan sistem pengembangan gamifikasi dengan menggunakan \textit{framework MDA} yang didasari domain pembelajaran spesifik. 
    \item Dalam proses pengembangannya, Aplikasi MedQ diuji menggunakan tiga pengujian. Tiga pengujian tersebut diantaranya:
    \begin{itemize}
        \item Pengujian fungsionalitas dilakukan menggunakan \textit{Black Box  Testing} yang menghasilkan 100\% “Berhasil” sehingga disimpulkan bahwa semua fitur berjalan sesuai dengan kriteria yang telah ditentukan. 
        \item Pengujian kegunaan menggunakan \textit{System Usability Scale} mendapatkan nilai 74,9 yang masuk ke dalam kategori \textit{Good}.
        \item Pengujian pengalaman pengguna menggunakan \textit{User Experience Questionnaire} menujukkan bahwa aplikasi MedQ termasuk dalam kategori \textit{Excellent} untuk aspek daya tarik, efisiensi, ketepatan, dan stimulasi, serta termasuk dalam kategori \textit{Good} untuk aspek kejelasan dan kebaruan. 
    \end{itemize}
    \item Elemen gamifikasi berhasil diimplementasikan ke dalam MedQ dinilai dari seluruh pengujian yang dilakukan pada penelitian ini
\end{enumerate}
\section{Saran}
Aplikasi ini masih memiliki beberapa kekurangan. Beberapa saran yang 
dapat dipertimbangkan sebagai solusi dari kekurangan aplikasi ini adalah sebagai 
berikut :
\begin{enumerate}
    \item Dapat ditambahkan fitur \textit{Tutorial} untuk lebih mempermudah pengguna
    \item Fitur materi masih bisa ditambahkan berdasarkan domain pembelajaran lain
\end{enumerate}