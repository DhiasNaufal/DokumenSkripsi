\textit{
	This thesis discusses the development of gamification in the Clinical Decision Support System (CDSS) learning application and the benefits of implementing gamification in the learning process. CDSS is a system used in the healthcare field to assist in more effective clinical decision-making. However, in the context of learning, the use of CDSS is still less engaging due to the complexity of combining medical and informatics knowledge.
The aim of this research is to develop gamification elements, such as rewards, challenges, and leaderboards, in the CDSS learning application. In this development process, data collection is conducted through surveys and interviews with students and educators to understand their needs and expectations regarding the use of gamification in learning. The functionality of the application was also tested using Black Box Testing, and usability and user experience were tested with 38 respondents.
The research results show that the implementation of gamification in the CDSS learning application can enhance student motivation and engagement. Gamification elements, such as rewards given as recognition for specific achievements, challenges that stimulate competition, and leaderboards that display participant rankings, can provide additional motivation for students to learn with enthusiasm and actively participate in the learning process.
The application of gamification in learning also provides other benefits, such as improving concept understanding, enhancing memory retention, and strengthening social interaction among students. By integrating gamification elements into the CDSS learning application, it is expected that the learning process will become more engaging, interactive, and effective in achieving learning objectives. This is evidenced by the SUS (System Usability Scale) test with a "Good" result, and the UEQ (User Experience Questionnaire) test with an "Excellent" result.
In conclusion, the development of gamification in the CDSS learning application has the potential to enhance student motivation and engagement. The importance of gamification in learning lies in its ability to enhance interaction, provide incentives, and create a more engaging and enjoyable learning experience. Therefore, the use of gamification can be an effective strategy in enhancing the effectiveness of learning in various fields, including learning using CDSS applications.
}

\noindent\textbf{Keywords} : \textit{Gamification}, \textit{Learning Application}, \textit{Clinical Decision Support System}, \textit{Motivation}, \textit{Interaction}