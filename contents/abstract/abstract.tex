\textit{
	This thesis discusses the development of gamification in the Clinical Decision Support System (CDSS) learning application and the benefits of implementing gamification in the learning process. In the context of education, CDSS is still uncommon and quite complex as it combines medical and informatics sciences.
	This research aims to develop an electronic learning media specifically focused on CDSS topics using the Feature-drawer Development method, as well as developing gamification elements such as rewards, challenges, and leaderboards in the learning application using the MDA Framework. The development process starts with the overall design of the application and determining the learning objectives, followed by designing suitable gamification elements into the overall application design.
	Subsequently, functional testing of the application is conducted using Black Box Testing, as well as usability testing using the System Usability Scale and user experience testing using the User Experience Questionnaire with 38 respondents.
	The research results will be in the form of a User Interface application, which represents the pages of each developed feature. There is also an Android application developed based on the previous design process. The functional testing results obtained 100\%, indicating that the application meets the expected criteria. For the System Usability Scale test, the average score obtained was 74.9, which falls into the "Good" category, and the User Experience Questionnaire yielded "Excellent" results.
	The testing results indicate that the implementation of gamification in the CDSS learning application can enhance students' motivation and engagement. Gamification elements, such as rewards given for specific achievements, challenges that foster competition, and leaderboards displaying participants' rankings, can provide additional incentives for students to learn more enthusiastically and actively participate in the learning process.
	In conclusion, the gamification development in the CDSS learning application has the potential to enhance students' motivation and engagement. The importance of gamification in learning lies in its ability to enhance interaction, provide incentives, and create a more engaging and enjoyable learning experience. Thus, the use of gamification can be an effective strategy in improving the effectiveness of learning in various fields, including learning using CDSS applications.
}

\noindent\textbf{Keywords}: \textit{Gamification}, \textit{Learning Application}, \textit{Clinical Decision Support System}, \textit{Motivation}, \textit{Interaction}