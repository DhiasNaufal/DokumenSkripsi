Skripsi ini membahas tentang pengembangan gamifikasi pada aplikasi pembelajaran \textit{Clinical Decision Support System}(CDSS) dan keuntungan penerapan gamifikasi dalam proses pembelajaran. 
Dalam konteks pendidikan, CDSS masih kurang umum dan cukup kompleks karena ilmu ini yang menggabungkan ilmu medis dan ilmu informatika.
Penelitian ini bertujuan untuk mengembangkan sebuah media pembelajaran elektronik yang secara spesifik mengambil topik CDSS menggunakan metode pengembangan \textit{Feature-drawer Development},
serta mengembangkan elemen-elemen gamifikasi, seperti pemberian hadiah, tantangan, dan \textit{leaderboard} dalam aplikasi pembelajaran menggunakan kerangka kerja \textit{MDA Framework}. 
Proses pengembangan dimulai dari proses desain keseluruhan aplikasi dan menentukan tujuan pembelajaran dilanjutkan dengan mendesain elemen gamifikasi yang cocok ke dalam desain keseluruhan aplikasi.
Setelah itu, dilanjutkan dengan proses pengujian fungsionalitas aplikasi menggunakan \textit{Black Box Testing}, serta pengujian kebergunaan menggunakan \textit{System Usability Scale} dan pengujian pengalaman pengguna menggunakan \textit{User Experience Questionnaire} dengan menggunakan responden sebanyak 38.
Hasil penelitian akan berupa sebuah \textit{User Interface} aplikasi yang merupakan halaman dari masing masing fitur yang dikembangkan. Ada juga sebuah aplikasi Android yang dikembangkan dari proses desain sebelumnya.
Hasil pengujian fungsionalitas mendapatkan hasil 100\% yang berarti kriteria aplikasi sesuai dengan yang diharapkan. 
Untuk pengujian \textit{System Usability Scale} mendapatkan rata rata skor 74,9 yang termasuk ke dalam kategori \textit{"Good"}, dan \textit{User Experience Questionnaire} dengan hasil \textit{Excellent}.
Hasil pengujian menunjukkan bahwa implementasi gamifikasi pada aplikasi pembelajaran \textit{CDSS} dapat meningkatkan motivasi dan keterlibatan peserta didik. Elemen-elemen gamifikasi, seperti hadiah yang diberikan sebagai penghargaan atas pencapaian tertentu, tantangan yang memacu kompetisi, dan leaderboard yang memperlihatkan peringkat peserta, mampu memberikan dorongan tambahan bagi peserta didik untuk belajar dengan lebih antusias dan berpartisipasi aktif dalam proses pembelajaran.
Dalam kesimpulannya, pengembangan gamifikasi pada aplikasi pembelajaran \textit{CDSS} memiliki potensi untuk meningkatkan motivasi dan keterlibatan peserta didik. Pentingnya gamifikasi dalam pembelajaran terletak pada kemampuannya untuk meningkatkan interaksi, memberikan insentif, dan menciptakan pengalaman pembelajaran yang lebih menarik dan menyenangkan. Dengan demikian, penggunaan gamifikasi dapat menjadi strategi yang efektif dalam meningkatkan efektivitas pembelajaran di berbagai bidang, termasuk pembelajaran menggunakan aplikasi \textit{CDSS}.

% \textit{CDSS} merupakan sistem yang digunakan dalam bidang kesehatan untuk membantu pengambilan keputusan klinis yang lebih efektif. 
% Dalam proses pengembangan ini, dilakukan pengumpulan data melalui survei dan wawancara dengan peserta didik dan tenaga pengajar untuk memahami kebutuhan dan harapan mereka terhadap penggunaan gamifikasi dalam pembelajaran.
% Dilakukan juga pengujian fungsionalitas aplikasi menggunakan \textit{Black Box Testing}, dan pengujian dengan menggunakan responden sebanyak 38 untuk mengujikan kebergunaan dan pengalaman pengguna.
% Hasil penelitian menunjukkan bahwa implementasi gamifikasi pada aplikasi pembelajaran \textit{CDSS} dapat meningkatkan motivasi dan keterlibatan peserta didik. Elemen-elemen gamifikasi, seperti hadiah yang diberikan sebagai penghargaan atas pencapaian tertentu, tantangan yang memacu kompetisi, dan leaderboard yang memperlihatkan peringkat peserta, mampu memberikan dorongan tambahan bagi peserta didik untuk belajar dengan lebih antusias dan berpartisipasi aktif dalam proses pembelajaran.
% Penerapan gamifikasi dalam pembelajaran juga memberikan manfaat lain, seperti peningkatan pemahaman konsep, meningkatkan daya ingat, dan memperkuat interaksi sosial antara peserta didik. Dengan mengintegrasikan elemen gamifikasi dalam aplikasi pembelajaran \textit{CDSS}, diharapkan proses pembelajaran menjadi lebih menarik, interaktif, dan efektif dalam mencapai tujuan pembelajaran. Ini dibuktikan dengan pengujian \textit{SUS} dengan hasil \textit{Good}, dan \textit{UEQ} dengan hasil \textit{Excellent}

\noindent{\textbf{Kata kunci}}: Gamifikasi, Aplikasi Pembelajaran, \textit{Clinical Decision Support System}, Motivasi, Keterlibatan

\vspace{1cm}

% %HAPUS YANG TIDAK PERLU
% %-------------------------------------------------
% \noindent\fbox{%
% 	\parbox{\textwidth}{%
% \textbf{Contoh Abstrak Teknik Elektro:} \\

% \hspace{1cm} "Penelitian ini bertujuan untuk mengembangkan sistem pengendalian suhu ruangan dengan menggunakan microcontroller. Metodologi yang digunakan adalah desain sirkuit, implementasi sistem pengendalian, dan pengujian performa. Hasil penelitian menunjukkan 
% bahwa sistem pengendalian suhu ruangan yang dikembangkan mampu mengendalikan suhu ruangan dengan akurasi sebesar ±0,5°C. Kesimpulan dari penelitian ini adalah sistem pengendalian suhu ruangan yang dikembangkan efektif dan efisien. \\

% Kata kunci: microcontroller, sistem pengendalian suhu, akurasi."
% \vspace{5mm}

% \textbf{Contoh Abstrak Teknik Biomedis:} \\

% \hspace{1cm} "Penelitian ini bertujuan untuk mengevaluasi keefektifan prototipe alat pemantau denyut jantung berbasis elektrokardiogram (ECG) untuk pasien jantung. Metodologi yang digunakan meliputi desain dan pembuatan prototipe, pengujian dengan pasien, dan analisis data. Hasil penelitian menunjukkan bahwa prototipe alat pemantau denyut jantung berbasis ECG memiliki 
% akurasi yang baik dan mampu memantau denyut jantung pasien secara efektif. Kesimpulan dari penelitian ini adalah prototipe alat pemantau denyut jantung berbasis ECG merupakan solusi 
% yang efektif dan efisien untuk memantau pasien jantung. \\

% Kata kunci: elektrokardiogram, alat pemantau denyut jantung, akurasi."
% \vspace{5mm}

% 	}%
% }

% %-------------------------------------------------

% \noindent\fbox{%
% 	\parbox{\textwidth}{%
% 		\textbf{Contoh Abstrak Teknologi Informasi:} \\
		
% \hspace{1cm} "Penelitian ini bertujuan untuk mengevaluasi keamanan dan privasi pengguna aplikasi media sosial terpopuler. Metodologi yang digunakan meliputi analisis kebijakan privasi dan pengaturan keamanan, pengujian penetrasi, dan survei pengguna. Hasil penelitian 
% menunjukkan bahwa beberapa aplikasi media sosial memiliki kebijakan privasi yang kurang jelas dan rendahnya tingkat keamanan. Kesimpulan dari penelitian ini adalah pentingnya meningkatkan kebijakan privasi dan tingkat keamanan pada aplikasi media sosial untuk melindungi privasi dan data pengguna. \\
		
% Kata kunci: media sosial, keamanan, privasi, pengguna."
% \vspace{5mm}
		
% 	}%
% }