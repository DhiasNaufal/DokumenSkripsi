\chapter{Tinjauan Pustaka dan Dasar Teori}

\section{Tinjauan Pustaka}
Bab ini akan membahas tinjauan pustaka yang mencakup penelitian-penelitian sebelumnya yang telah dilakukan.
Penelitian-penelitian ini akan dijelaskan sebagai referensi penulis dalam menyelesaikan Tugas Akhir.
Selain itu, bab ini akan menjelaskan teori-teori yang menjadi dasar dalam pembuatan Tugas Akhir ini.
\subsection{Judul 1}
\section{Analisis Perbandingan Metode}
\section{Dasar Teori}
\subsection{Media Pembelajaran}
\subsection{Teori Game}
\subsubsection{Elemental Tetrad}
Elemental Tetrad adalah sebuah kerangka konseptual yang digunakan dalam desain dan analisis produk atau layanan digital. 
Konsep ini pertama kali diperkenalkan oleh Jesse James Garrett, seorang desainer pengalaman pengguna terkemuka, 
dan berfokus pada empat elemen utama yang saling berinteraksi dalam pengalaman pengguna digital.

\subsubsection{The MDA Framework}
MDA (Mechanics, Dynamics, Aesthetics) Framework adalah sebuah kerangka kerja yang digunakan dalam pengembangan permainan
(game development) untuk menganalisis dan memahami elemen-elemen inti yang membentuk pengalaman bermain game. 
Konsep ini pertama kali diperkenalkan oleh Robin Hunicke, Marc LeBlanc, dan Robert Zubek pada tahun 2004.
\subsubsection{The Game Design Spiral}
Game Design Spiral (lingkaran desain permainan) adalah pendekatan iteratif 
dalam desain permainan yang menggabungkan siklus pengembangan dan pengujian berulang untuk menciptakan 
permainan yang lebih baik seiring berjalannya waktu. Pendekatan ini memungkinkan desainer permainan untuk 
memperbaiki dan meningkatkan desain mereka melalui siklus yang terus berulang.
\subsection{Gamifikasi}
\subsection{FDD}
\subsection{Black Box Testing}
\subsection{\textit{System Usability Testing}(SUS)}
\subsection{\textit{User Experience Questionnaire}(UEQ)}


% Di dalam tinjauan pustaka hasil akhirnya adalah analisis secara kualitatif atau pun secara kuantitatif kelebihan dan kekurangan metode jika dikaitkan dengan masalah, batasan-batasan masalah dan solusi yang dinginkan.
% Analisis kuantitatif tidak wajib teapi mempunyai nilai tambah di dalam tugas akhir saudara. Bagian ini menjelaskan kenapa metode tersebut dipilih dan uraikan dengan lebih jelas metode pelaksanaan tugas akhir yang ingin Anda lakukan. 
% \section{Pertanyaan Tugas Akhir (Jika Perlu)}

% Pertanyaan tugas akhir bersifat opsional dan dapat ditambahkan untuk menekankan hal-hal yang hendak diketahui dari tugas akhir berdasar pada tujuan tugas akhir. Pertanyaan tugas akhir dikenal dengan RQ (\textit{Research Question}) dan harus memiliki keterkaitan dengan RO (\textit{Research Objective}). Satu RO dapat memiliki satu atau lebih dari satu RQ. 

