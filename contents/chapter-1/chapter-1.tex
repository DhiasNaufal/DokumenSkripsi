\chapter{Pendahuluan}
Pada bab pendahuluan ini akan dipaparkan mengenai Latar Belakang, Rumusan Masalah, dan Batasan Masalah dalam menentukan pengerjaan Tugas Akhir.
Kemudian dijelaskan juga mengenai Tujuan dari dibuatnya tugas akhir ini, dan Manfaat yang akan diperoleh dari hasil akhir.
Setelah itu, di akhir bab ini akan dijelaskan mengenai sistematika penulisan sebagai gambaran umum mengenai isi dari tugas akhir ini.

\section{Latar Belakang}
%========= Paragraf 1 : Pembahasan CDSS sebagai Subject Media Pembelajaran yang akan dibuat=======================
Sistem Diagnosis berbasis Pendukung Keputusan (SDBPK) atau yang lebih dikenal dengan \textit{Clinical Decision Support System (CDSS)}
merupakan sebuah sistem komputer yang dibuat spesifik untuk membantu tenaga kesahatan.
Sistem ini merupakan inovasi terkini dalam dunia medis yang bertujuan untuk meningkatkan kualitas perawatan medis dan pengambilan keputusan.
Pengembangan dan penelitian terus dilakukan guna meningkatkan efektivitas sistem ini dalam mendukung pengambilan keputusan yang optimal.
\textit{Clinical Decision Support Systems (CDSS)} adalah sistem yang muncul dari gabungan ilmu informatika dan ilmu pengetahuan medis.
Dalam \textit{CDSS}, ilmu informatika memberikan kerangka teknologi dan pemrosesan data yang diperlukan untuk mengumpulkan, mengelola, dan menganalisis informasi kesehatan.
Di sisi lain, ilmu pengetahuan medis memberikan dasar pengetahuan tentang penyakit, diagnosis, pengobatan, dan pedoman klinis yang digunakan dalam sistem. 
Dengan menggabungkan kedua ilmu ini, \textit{CDSS} dapat menyediakan informasi yang tepat waktu dan akurat kepada profesional medis, membantu mereka dalam membuat keputusan klinis yang lebih baik dan berdasarkan bukti.
\textit{CDSS} mampu mengintegrasikan data pasien, penelitian medis terbaru, serta algoritma kecerdasan buatan untuk memberikan rekomendasi perawatan yang lebih efektif dan individual.
Dalam konteks pendidikan, ilmu \textit{CDSS} masih perlu banyak peningkatan.
Mahasiswa atau tenaga medis yang belajar tentang \textit{CDSS} seringkali menghadapi kendala dalam memahami konsep dan mengaplikasikan pengetahuan \textit{CDSS} secara praktis.
Kurangnya sumber daya pembelajaran yang interaktif, terstruktur, dan berfokus pada aplikasi \textit{CDSS} dapat menjadi hambatan dalam pembelajaran efektif tentang sistem ini.
Dalam rangka meningkatan pembelajaran CDSS, diperlukan sebuah media pembelajaran yang efektif dan berfokus pada aplikasi \textit{CDSS}.
Media pembelajaran yang tepat dapat membantu mengatasi kendala yang dihadapi oleh mahasiswa atau tenaga medis dalam memahami konsep dan menerapkan pengetahuan \textit{CDSS} secara praktis.

Teknologi Informasi mengalami perkembangan yang sangat pesat, berbanding lurus dengan beragamnya pemanfaatan Teknologi Informasi dalam konteks pendidikan.
Salah satunya ialah hadirnya media pembelajaran yang mengadopsi Teknologi Informasi untuk mempermudah manusia dalam penyampaian informasi pembelajaran.
Media pembelajaran sendiri adalah sebuah medium yang memuat informasi atau pesan instruksional dan dapat digunakan dalam proses pembelajaran.
Media pembelajaran merupakan elemen penting bagi peserta didik untuk membantu memperoleh konsep baru, keterampilan dan kompetensi.
Dengan memanfaatkan Media pembelajaran yang tepat, akan membantu peserta didik untuk meningkatkan interaksi dalam proses pembelajaran 
sehingga peserta didik tidak akan merasa bosan dalam pembelajaran\cite{hasan2021media}.
Media pembelajaran yang memanfaatkan teknologi informasi disebut dengan media pembelajaran elektronik.
Media pembelajaran elektronik mencakup berbagai bentuk, mulai dari aplikasi mobile, platform pembelajaran online, hingga simulasi interaktif. 
Keberadaan media pembelajaran elektronik memberikan potensi yang besar dalam meningkatkan efektivitas pembelajaran, menghadirkan pengalaman belajar yang interaktif, dan meningkatkan keterlibatan peserta didik.
Melalui media pembelajaran elektronik, peserta didik dapat mengakses berbagai konten pembelajaran secara fleksibel, baik melalui perangkat komputer, tablet, maupun smartphone.
Dalam konteks ini, media pembelajaran elektronik memberikan keleluasaan bagi peserta didik untuk belajar kapan saja dan di mana saja, tanpa terbatas oleh waktu dan tempat. 
Penggunaan media pembelajaran elektronik juga dapat meningkatkan keterlibatan peserta didik dalam proses pembelajaran.
Dengan tampilan yang menarik dan interaktif, media pembelajaran elektronik dapat membangkitkan minat dan motivasi peserta didik dalam mempelajari materi pembelajaran. 
Salah satu strategi yang efektif untuk menjaga minat dan motivasi peserta didik dalam pembelajaran adalah dengan menggabungkan konsep permainan ke dalam media pembelajaran.
Pendekatan ini dikenal sebagai Gamifikasi Pembelajaran. Dalam Gamifikasi Pembelajaran, elemen-elemen permainan seperti poin, level, tantangan, hadiah, dan peringkat digunakan untuk menciptakan pengalaman yang menarik dan menyenangkan bagi peserta didik.
Tujuan utama dari penggunaan Gamifikasi Pembelajaran adalah untuk mendorong peserta didik agar lebih aktif, terlibat, dan bersemangat dalam proses pembelajaran.
Dengan memanfaatkan elemen permainan, seperti sistem poin yang memotivasi pencapaian, tingkatan yang memberikan tantangan bertahap, dan hadiah yang memberikan pengakuan atas prestasi, peserta didik akan merasa lebih termotivasi untuk mengikuti dan menyelesaikan tugas pembelajaran.

% Dalam penelitian ini, penulis akan membahas proses pengembangan desain gamifikasi pada sebuah media pembelajaran elektronik yang akan dikembangkan.
% Dengan menggunakan elemen-elemen seperti pemberian poin, tingkat, pencapaian, kompetisi, dan hadiah dalam aplikasi pembelajaran e-learning, diharapkan pendekatan ini dapat memicu minat, antusiasme, dan keterlibatan siswa atau mahasiswa.

\newpage
\section{Rumusan Masalah}
% Efektivitas pembeljaran menjadi sebuah poin utama dalam mempelajari sebuah materi.
Salah satu upaya untuk meningkatkan efektivitas pembelajaran pada materi tertentu adalah dengan menyediakan media pembelajaran khusus yang secara spesifik membahas materi tersebut.
Dalam konteks pendidikan, sistem \textit{Clinical Decision Support System} masih belum berkembang dengan baik.
Hal tersebut dapat dilihat dari belum adanya media pembelajaran spesifik yang membahas mengenai sistem tersebut.
Hadirnya media pembelajaran elektronik meruapakan peluang besar dalam pendidikan untuk meningkatkan kulitas dan efektifitas pembelajaran.
Dengan demikian, masalah pertama dapat dirumuskan sebagai berikut:
\begin{list}{\small$\bullet$}{}
	\item Hingga saat ini, belum tersedia media pembelajaran elektronik yang secara khusus membahas tentang sistem \textit{Clinical Decision Support System}
\end{list}

Tidak sampai di situ, ada pula tantangan dari sebuah media pembelajaran elektronik untuk menarik perhatian siswa atau pengguna untuk menggunakannya.
Media pembelajaran yang monoton tidak akan membantu meningkatkan efektivitas dan kualitas. Salah satu metode yang sudah hadir saat ini ialah
pengadopsian konsep permaianan ke dalam sebuah media pembelajaran elektronik. Maka, masalah selanjutnya dapat dirumuskan sebagai berikut:
\begin{list}{\small$\bullet$}{}
	\item Bagaimana cara mengadaptasi konsep permainan ke dalam sebuah media pembelajaran elektronik yang cocok untuk meningkatkan motivasi dan keterlibatan?
	\item Bagaimana implementasi konsep gamifikasi dalam media pembelajaran elektronik dapat mempengaruhi motivasi dan keterlibatan peserta didik dalam pembelajaran?
\end{list}


%-------------------------------------------------	

\section{Tujuan Penelitian}

Tujuan dari pengembangan tugas akhir ini ialah :
\begin{enumerate}
	\item Menciptakan sebuah media pembelajaran elektronik yang efektif yang secara spesifik membahas mengenai sistem \textit{Clinical Decision Support System}
	\item Mengintegrasikan konsep permainan yang cocok dalam media pembelajaran elektronik guna menciptakan pengalaman pembelajaran yang berkualitas dan meningkatkan motivasi.
	\item Menguji fungsionalitas, kegunaan, dan pengalaman pengguna dari media pembelajaran elektronik kepada calon penggunanya.
\end{enumerate}
\newpage
\section{Batasan Penelitian}
Berdasarkan keterbatasan waktu dan sumber daya manusia, pembahasan yang terdapat pada Tugas Akhir ini memiliki beberapa batasan, diantaranya ialah :
\begin{enumerate}
\item Objek penelitian: Penelitian ini berfokus pada desain dan pengembangan Gamifikasi pada sebuah Applikasi pembelajaran.
\item Metode penelitian: Penelitan Desain dan Pengembangan dengan menggunakan metode Feature Driven Development
\item Waktu dan tempat penelitian: Penelitian ini berlangsung dari ---------
\item Populasi dan sampel:Penelitian ini mengikutsertakan mahasiswa Teknik Biomedis Departemen Teknik Elektro dan Teknologi Informasi sebagai sampel pengujian Pengalaman User untuk aplikasi yang dikembangkan.
\item Variabel: Vriabel bebasnya adalah Media Pembelajaran, dan variabel terikatnya adalah efektivitas dan efisiensi.
\item Hipotesis: Pengimplementasian Gamifikasi pada Aplikasi Pembelajaran dapat mempengaruhi efektivitas dan motivasi
\item Keterbatasan Penelitian: Aplikasi yang dikembangkan hanya dapat berjalan di Sistem Operasi Android. Proses pengujian pengalaman pengguna menggunakan responden terbatas.
\end{enumerate}

\section{Manfaat Penelitian}
Pengembangan aplikasi pembelajaran ini diharapkan dapat memberikan manfaat dengan memperkenalkan sistem pembantu keputusan yang khusus digunakan dalam konteks kesehatan atau bidang medis.
Selain itu dengan pengadopsian gamifikasi dalam aplikasi pembelajarannya diharapkan dapat meningkatkan motivasi pengguna dan efektivitas pengguna dalam mempelajari materi.

\section{Sistematika Penulisan}
\begin{enumerate}
	\item Bab I Mengurai dan menjelaskan tentang latar belakang, rumusan 
	masalah yang akan dijawab pada penelitian ini, batasan masalah yang membatasi 
	pelaksanaan dari penelitian ini, tujuan yang akan dicapai pada penelitian ini, serta manfaat 
	penelitian bagi pihak-pihak terkait.
	\item Bab II akan menyajikan ulasan literatur berdasarkan penerapan gamifikasi pada sistem telah ada sebelumnya yang menjadi dasar dan teori pendukung dalam pengembangan aplikasi ini.
	Selain itu, juga terdapat penjelasan tentang teori-teori yang menjadi dasar dalam pembuatan aplikasi tugas akhir ini, 
	termasuk bahasa pemrograman dan perangkat lunak yang digunakan oleh penulis.
	\item BAB III akan memuat informasi mengenai persyaratan yang diperlukan oleh penulis dalam pengembangan aplikasi tugas akhir ini.
	Hal-hal tersebut meliputi penjelasan rinci tentang perangkat lunak dan perangkat keras yang digunakan oleh penulis, 
	serta urutan langkah dalam pembuatan aplikasi mulai dari penerapan metode \textit{Feature-Driven Development} sebagai panduan metode pengembangan, hingga tahap pengujian.
	\item BAB IV memuat penjelasan terperinci mengenai output atau hasil dari aplikasi yang telah dibuat oleh penulis.
	Seluruh proses pengembangan, mulai dari tahap awal hingga mencapai tingkat kesiapan aplikasi yang siap digunakan, serta hasil pengujian, 
	disajikan dalam bentuk tangkapan layar \textit{(screenshot)} yang dilengkapi dengan deskripsi penjelasan untuk setiap tahapnya.
	\item BAB V berisi rangkuman dari seluruh proses pembuatan aplikasi, yaitu solusi yang dianggap dapat mengatasi setiap rumusan masalah dalam tugas akhir ini.
	Di samping itu, juga terdapat rekomendasi dan langkah-langkah yang dapat diambil untuk pengembangan aplikasi ini agar menjadi lebih baik dan dapat digunakan sebagai sumber pembelajaran dalam konteks pembahasan yang lebih lanjut.
\end{enumerate}

