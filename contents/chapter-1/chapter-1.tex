\chapter{Pendahuluan}
Pada bab pendahuluan ini akan dipaparkan mengenai Latar Belakang, Rumusan Masalah, dan Batasan Masalah dalam menentukan pengerjaan Tugas Akhir.
Kemudian dijelaskan juga mengenai Tujuan dari dibuatnya tugas akhir ini, dan Manfaat yang akan diperoleh dari hasil akhir.
Setelah itu, di akhir bab ini akan dijelaskan mengenai sistematika penulisan sebagai gambaran umum mengenai isi dari tugas akhir ini.

\section{Latar Belakang}
%========= Paragraf 1 : Pembahasan CDSS sebagai Subject Media Pembelajaran yang akan dibuat=======================
Sistem Diagnosis berbasis Pendukung Keputusan (SDBPK) atau yang lebih dikenal dengan \textit{Clinical Decision Supoort System (CDSS)}
merupakan sistem komputer yang dibuat spesifik untuk membantu tenaga kesahatan.
Sistem ini merupakan sebuah terobosan dalam dunia medis yang dapat meningkatkan kualitas perawatan medis dalam pengambilan keputusan.
Sistem ini penting untuk dipelajari

Untuk meningkatkan kesadaran akan adanya sistem ini, 
Importance of CDSS to Public

Teknologi Informasi mengalami perkembangan yang sangat pesat, berbanding lurus dengan beragamnya pemanfaatan Teknologi Informasi dalam konteks penyampaian informasi.
Salah satunya ialah hadirnya media pembelajaran yang mengadopsi Teknologi Informasi untuk mempermudah manusia dalam penyampaian informasi pembelajaran.
Media pembelajaran sendiri adalah sebuah medium yang memuat informasi atau pesan instruksional dan dapat digunakan dala proses pembelajaran.
Media pembelajaran merupakan elemen yang cukup penting bagi peserta didik untuk membantu memperoleh konsep baru, keterampilan dan kompetensi.
Dengan memanfaatkan Media pembelajaran yang tepat, akan membantu peserta didik untuk meningkatkan interaksi dalam proses pembelajaran 
sehingga peserta didik tidak akan merasa bosan dalam pembelajaran\cite{hasan2021media}.

Media pembelajaran yang mengadopsi Teknologi informasi dikenal juga dengan E-Learning, sekarang sudah ada M-Learning dimana M-learning dibandingkan E-Learning baisa blablabla

Goals dari masalah yang dihadapi ialah membuat suatu media pembelajaran yang tidak monoton dan lebih menarik (dapat meningkatkan motvasi).
Gamifikasi merupakan metode blablabala
% menjanjikan akan meningkatnya kualitas perawatan klinis dengan membantu tenaga medis
% untuk mengakses dan menerapkan pengetahuan klinis secara relevan dan efisien, dengan demikian dapat mengambil keputusan yang lebih baik
% sesuai dengan perkembangan terbaru dalam bidang medis.


% Tidak semua orang mengetahui seberapa kuat sistem ini akan membantu kita untuk mengambil sebuah keputusan medis.


% Sistem tersebut dipelajari secara khusus pada program studi Teknik Biomedis Departemen Teknik Elektro dan Teknologi Informasi (DTETI) Universitas Gadjah Mada.
% %========= Ngejelasin akar Masalah ==========
% Berdasarkan hasil wawancara yang dilakukan dengan mahasiswa Teknik Biomedis yang sudah pernah mengambil mata kuliah tersebut, 
% mereka menyampaikan bahwa mereka mengalami kesulitan dalam memahami inti dari mata kuliah tersebut.
% Selain karena mata kuliah ini tergolong baru, 
% Faktor kesulitan tersebut tidak hanya disebabkan oleh sifat baru mata kuliah ini, 
% tetapi juga karena adanya dua core materi yang dianggap kompleks. 
% Core materi pertama adalah pengetahuan tentang Sistem Pembantu Keputusan (SPK) yang dipelajari oleh mahasiswa Teknologi Informasi, 
% sedangkan core materi kedua adalah pengetahuan tentang Diagnosis yang dipelajari oleh mahasiswa dari Kluster Kesehatan.
% Karena kompleksitas materi yang perlu dipahami, metode penyampaian yang efektif merupakan kunci dalam proses pembelajaran.

% Pendidikan merupakan salah satu aspek dalam kehidupan manusia yang memainkan peran penting dalam mengembangkan sebuah individu dan masyarakat.
% Melalui pendidikan, Individu akan memperoleh pengetahuan, keterampilan dan nilai-nilai positif.
% Nilai-nilai tersebut dapat diperoleh melalui sebuah proses pembelajaran yang dilakukan oleh seorang individu.
% Proses Pembelajaran dapat melibatkan dua pihak yaitu siswa sebagai pebelajar dan guru sebagai fasilitator, yang terpenting dalam kegiatan pembelajaran adalah terjadinya proses belajar \textit{(Learning Process)}.
% Salah satu elemen penting dalam sebuauh proses pembelajaran ialah adanya media yang menjadi sarana penyampaian informasi yang dilakukan antar
% Namun, dalam proses pelaksanaannya sering kali siswa atau mahasiswa mengalami kesulitan dalam mempertahankan motivasi dan minat dalam proses pembelajaran.
% Salah satu faktor yang memicu masalah tersebut ialah kurangnya ketertarikan dan keterlibatan mahasiswa terhadap materi yang disajikan dengan cara konvensional.
% Cara konvensional yang dimaksud merupakan pembelajaran yang berpusat pada guru, dimana guru berperan dalam mengendalikan penyajian pembelajaran.
% Cara ini juga diyakini masih belum efektif dikarenakan memerlukan kehadiran dua pihak dalam pelaksanaanya.
% Murid tidak dapat ikut andil atau pasif dalam proses penmbelajarannya kecuali kedua belah pihak bertemu pada sebuah pertemuan.

% %========== Paragraf 2 : Pembahasan adopsi teknologi pada Pendidikan =================== kata kunci %Perkembangan Teknologi -> Dibutuhkan media pembalajaran yang menarik
% Pengadopsian Teknologi Informasi merupakan salah satu upaya untuk mengatasi tantangan dalam pembelajaran konvensional tersebut.
% Salah satu solusi yang menarik ialah penggunakan teknologi e-learning atau pembelajaran elektronik.
% E-learning sendiri merupakan sebuah metode pendekatan pembelajaran dengan mengadopsi Teknolog Informasi dalam prosesnya, khususnya internet, sebagai media utama untuk menyampaikan materi pembelajaran kepada siswa atau mahasiswa. 


% Penerapan e-learning memberikan banyak manfaat untuk proses pembelajaran.
% Dengan penerapan teknologi ini, siswa atau mahasiswa akan memiliki fleksibilitas dalam mengatur waktu dan tempat belajar.
% Mereka dapat mengakses materi pembelajaran di mana saja dan kapan saja selama terhubung ke internet.
% Hal ini tergolong lebih efektif karena memungkinkan adanya pembelajaran jarak jauh, mengatasi keterbatasan geografis dan jadwal yang kaku.
% Selain itu, e-learning dapat meningkatkan kemungkinan interaksi yang lebih aktif antara siswa atau mahasiswa dengan materi pembelajaran.
% Melalui fitur-fitur seperti forum diskusi, kuis online, dan tugas daring, siswa atau mahasiswa dapat berpartisipasi secara aktif dalam proses pembelajaran.

% Meskipun e-learing telah memberikan solusi dalam hal fleksibilitas dan efektivitas interaksi, masih ada tantangan dalam mempertahankan motivasi dan keterlibatan siswa atau mahasiswa.
% Dalam hal ini, penerapan desain gamifikasi pada sebuah media pembelajaran dapat menjadi salah satu solusi yang diyakini efektif untuk meningkatkan motivasi dan keterlibatkan siswa atau mahasiswa.
% Gamifikasi merupakan sebuah pendekatan yang mengadaptasi elemen-elemen permainan atau game design dalam konteks non-permainan untuk meningkatkan motivasi dan partisipasi siswa atau mahasiswa. 

Dalam penelitian ini, penulis akan membahas proses pengembangan desain gamifikasi pada sebuah media pembelajaran elektronik yang akan dikembangkan.
Dengan menggunakan elemen-elemen seperti pemberian poin, tingkat, pencapaian, kompetisi, dan hadiah dalam aplikasi pembelajaran e-learning, diharapkan pendekatan ini dapat memicu minat, antusiasme, dan keterlibatan siswa atau mahasiswa.


\section{Rumusan Masalah}
Berdasarkan pemaparan pada bagian latar belakang, maka dapat dirumuskan masalah sebagai berikut :
\begin{enumerate}
	\item Proses pembelajaran pada Mata Kuliah SDBPK masih belum optimal dikarenakan merupakan sebuah mata kuliah yang baru dan cukup kompleks
	\item Mata Kuliah SDBPK di Departemen Teknik Elektro dan Teknologi Informasi (DTETI) belum memiliki media pembelajaran yang cocok untuk mahasiswanya
\end{enumerate}

%-------------------------------------------------	

\section{Tujuan Penelitian}

Tujuan dari pengembangan tugas akhir ini ialah :
\begin{enumerate}
	\item Mengembangkan media pembelajaran untuk mata kuliah Sistem Diagnosis Berbasis Pembantu Keputusan\textit{(SDBPK)} dengan adaptasi Gamifikasi untuk menciptakan media pembelajaran yang interaktif dan menarik
	\item Menguji fungsionalitas, kegunaan, dan pengalaman pengguna dari aplikasi pembelajaran kepada calon penggunanya.
\end{enumerate}
% \newpage 	
% \vspace{5mm}
% \begin{minipage}{0.92\textwidth}
% Berikut adalah beberapa contoh tujuan penelitian yang sesuai dengan topik 
% “perbaikan efisiensi penghematan energi pada sistem pencahayaan rumah tangga 
% melalui implementasi teknologi kontrol otomatis”:
% \end{minipage}
\section{Batasan Penelitian}
Berdasarkan keterbatasan waktu pengembangan dan sumber daya manusia, pembahasan yang terdapat pada Tugas Akhir ini memiliki beberapa batasan, diantaranya ialah :
\begin{enumerate}
\item Objek penelitian: spesifikasikan objek penelitian dan bidang teknik yang diteliti, misalnya teknik elektro, teknik biomedis, atau teknologi informasi.
% \item Objek Penelitian: Analisis perbandingan efektivitas dan efisiensi antara sistem manajemen proyek tradisional dan sistem manajemen proyek berbasis teknologi informasi. 
\item Metode penelitian: jelaskan metode penelitian yang akan digunakan dan 
bagaimana itu membatasi area penelitian, misalnya metode eksperimental, 
analisis simulasi, dll.
% \item Metode Penelitian: Penelitian kualitatif dengan menggunakan wawancara dan 
% survei terhadap para pelaku proyek di berbagai perusahaan. 
\item Waktu dan tempat penelitian: batasi waktu dan tempat penelitian, 
misalnya studi kasus pada tahun tertentu atau wilayah tertentu.
% \item Waktu dan Tempat Penelitian: Waktu penelitian adalah Januari-Juni 2022 di 
% perusahaan-perusahaan di wilayah Bantul. 
\item Populasi dan sampel: jelaskan populasi dan sampel yang akan diteliti, misalnya produk, mesin, atau sistem.
% \item Populasi dan Sampel: Populasi nya adalah perusahaan yang melakukan proyek, dan sampel diambil sebanyak 10 perusahaan yang menerapkan sistem manajemen 
% proyek tradisional dan 10 perusahaan yang menggunakan sistem manajemen 
% proyek berbasis teknologi informasi. 
\item Variabel dan hipotesis: jelaskan variabel yang akan diteliti dan hipotesis yang akan dibuktikan atau ditolak.
% \item Hipotesis: bahwa sistem manajemen proyek berbasis teknologi informasi lebih efektif dan efisien dibandingkan dengan sistem manajemen proyek tradisional.
\item Keterbatasan Penelitian: Keterbatasan penelitian adalah hambatan yang muncul dalam proses penelitian, seperti sumber daya yang terbatas, waktu, metodologi, dan kemampuan peneliti, yang mempengaruhi hasil dan validitas dari hasil penelitian.
% \item Keterbatasan Penelitian: Keterbatasan penelitian adalah penelitian hanya dilakukan pada perusahaan di wilayah Bantul dan hanya melibatkan wawancara dan survei sebagai metode pengumpulan data.
\end{enumerate}

\section{Manfaat Penelitian}
Dengan dikembangkannya Tugas Akhir ini diharapakan akan membawa manfaat kepada Mahasiswa DTETI

\begin{enumerate}
	\item Diaharapkan dengan dikembangkannya aplikasi ini dapat bermanfaat dalam kegiatan belajar mengajar Mahasiswa Departemen Teknik Elektro dan Teknologi Informasi 
	\item Diharapkan juga Tugas Akhir ini menjadi sebuah media pembelajaran yang dapat meningkatkan motivasi dalam pembelajaran
\end{enumerate}

\section{Sistematika Penulisan}
\begin{enumerate}
	\item Bab I Mengurai dan menjelaskan tentang latar belakang, rumusan 
	masalah yang akan dijawab pada penelitian ini, batasan masalah yang membatasi 
	pelaksanaan dari penelitian ini, tujuan yang akan dicapai pada penelitian ini, serta manfaat 
	penelitian bagi pihak-pihak terkait.
	
	\item Bab II berisi tentang metodologi penelitian yang terdiri dari desain penelitian, sumber data, Teknik pengumpulan data dan Teknik analisis data.
	\item BAB III berisi
	\item BAB IV berisi
	\item BAB V berisi
\end{enumerate}

